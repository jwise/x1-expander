\documentclass{article}
\usepackage{fontspec}
\usepackage{multicol}
\usepackage{enumitem}
\usepackage[paperheight=7in,paperwidth=5in,margin=0.3in,heightrounded]{geometry}
\usepackage[ngerman]{babel}

\setlength{\parindent}{0in}
\setlength{\parskip}{8pt}
\setlength{\itemsep}{0in}
\setlength{\topsep}{0in}
\setlength{\tabcolsep}{0in}

\setmainfont{Futura Std}

\begin{document}

% \fontspec{Futura Std}
% \fontsize{9}{13.9}
% \selectfont
\fontsize{10}{10}\selectfont

\textbf{Thank you for buying X1Plus Expander!}  Included in this box, you will find:

\fontsize{9}{9}\selectfont
\begin{multicols}{2}
\begin{itemize}
\item{This document.}
\vspace{-3pt}
\item{1x X1P-001 wiring harness.}
\vspace{-3pt}
\item{1x X1P-002 mainboard.}
\vspace{-3pt}
\item{3x M3 x 12mm screws.}
\vspace{-3pt}
\item{1x M3 x 18mm screw.}
\vspace{-3pt}
\item{1x ping-pong ball.}
\vspace{-3pt}
\end{itemize}
\end{multicols}
\fontsize{10}{10}\selectfont

\vspace{-8pt}

First, take a moment to make sure you have it all (especially the ping-pong
ball, which Peanut, the cat, told me was very important to include), and
have a read through the Instructions for Safe Use below (or, if you prefer,
in English on the other side).  Then, scan the QR
code on the box, or go to https://accelerated.tech/expander/, to get started
installing your Expander into your printer.

\hspace*{3in}Happy printing! \\
\hspace*{3in}joshua (and Peanut)

\begin{center}
\fontsize{14}{14}\selectfont
\textbf{Sicherheitshinweise}
\end{center}

\fontsize{7}{7}\selectfont

Der X1Plus Expander ist nicht für sicherheitskritische Anwendungen vorgesehen. Verwendungen wie der Betrieb einer Druckkammerheizung im geschlossenen Regelkreis setzen unabhängige Betriebssicherheitsmaßnahmen voraus. 

Für Veränderungen am Gerät durch herstellerfremde Hardware kann keine Haftung übernommen werden. Unbeaufsichtigte Druckvorgänge sind in keinem Fall empfohlen. 

Maximale Leistungsaufnahmen aus den integrierten Stromversorgungen sind 2A auf dem 5V-Ausgang oder 1A auf dem 3.3V-Ausgang, wobei eine Gesamtleistungsaufnahme von 10W nicht überschritten werden darf. Die Steckverbinder Strom Grenzwerte sind den entsprechenden Datenblättern zu entnehmen.

Der X1Plus Expander darf zum Betrieb ausschließlich am Netzteil des X1 Carbon angeschlossen sein. 

Der X1Plus Expander muss in einem Gehäuse aus nicht leitendem Material montiert werden und darf nicht mit leitenden Materialien in Verbindung kommen.

Das Anschließen nicht zugelassener Geräte an jegliche Steckverbinder kann das Gerät beschädigen oder zu Einschränkungen der Funktionen beitragen und kann zum Erlöschen von Garantieansprüchen führen. 

Der X1Plus Expander ist nicht wasserdicht, spritzwassergeschützt oder zum Einsatz in feuchten oder heißen Umgebungen geeignet. Er darf nicht auf nassen oder leitfähigen Oberflächen abgelegt werden, solange eine Netzverbindung besteht. Ebenso darf er nicht im Inneren der Druckkammer und im Arbeitsbereich des Druckers platziert werden. 

Um Beschädigungen an der Leiterplatte zu vermeiden, ist Vorsicht bei der Handhabung geboten. Arbeiten dürfen nur nach Unterbrechung der Netzverbindung vorgenommen werden. Um das Risiko der Beschädigung durch elektrostatische Entladung oder mechanische Einwirkung zu vermeiden, sollte die Leiterplatte nur an den Außenkanten berührt werden, nicht an Lötstellen oder Komponenten. 

Alle eventuell verwendeten Peripheriegeräte wie etwa Zusatzmodule für den X1Plus Expander müssen die geltenden Sicherheitsvorschriften des Landes erfüllen und entsprechend gekennzeichnet sein.



\enddocument
